\chapter{Host company and Project context}

\renewcommand{\chaptername}{Chapter}

\section*{Introduction}

This chapter is reserved to present STILL GmbH as the host company, its organizational structure, the mother 
company KION group. It will then proceed to describe the range of products that the company produces.
The second part is dedicated to set the project context by explaining the problem statement, the motivation 
behind this thesis project, and its specifications.
The final part will emphasize the work methodology adopted to carry out this project.

\begin{sloppypar}
\section{Host company: STILL GmbH}
\end{sloppypar}

This section introduces the host group and company through their activities, products, and activities

\subsection{KION Group and STILL GmbH}

STILL GmbH, based in Hamburg, Germany, is a leading manufacturer of intralogistics solutions with 14 locations 
in Germany and a global sales network spanning 246 locations. 
Operating under the KION Group, Europe’s largest forklift truck manufacturer, STILL boasts over 100 years of 
experience. The company develops highly efficient, client-tailored products, serving businesses of all sizes 
with a wide range of forklift trucks—from manually driven forklifts to high-reach trucks and fully automated 
vehicles—alongside consultancy services and software solutions. 

STILL prioritizes smart logistics and energy optimization while maintaining award-winning product quality, 
catering to industries such as food and retail, automotive, and electronics. Employing over 9,000 people across 
departments like sales and marketing, research and development, production, mechatronics, and quality assurance, 
STILL remains at the forefront of intralogistics innovation. 

KION Group is one of the global leaders in the fields of industrial trucks and supply chain solutions.
It is the mother company of: Linde, Dematic Baoli, OM, Fenwick, and STILL who produce the goods and services of the group as 
detailed in Figure \ref{KION Segments}. 

Present in 4 continents and hiring more than 42000 employees, KION's startegy is to ensure profitable ans sustainable growth 
while focusing on Automation and robotics deplyment as one of the main leaders of this growth. 

\begin{figure}[H]
    \begin{center}
     % Requires \usepackage{graphicx}
    \includegraphics[width=5in]{images/Chap0/KION_Segments.jpg}\\
    \caption{KION segment services and companies \cite{R1}}
    \label{KION Segments}
    \end{center}
    \end{figure}

    
\subsection{KION Management Hierarchy}

The company is composed of departments managing the operations in all companies that are divided by scope of 
interest like R\&D, Management, finances, etc.. Figure \ref{KION Hierarchy} illustrates the different areas of 
responsibility of the Executive
Board. The Autonomous vehicles team belongs to the Mobile Automation department under CTO. 

\subsection{STILL Products}

The 2017-established Autonomous vehicles team aims to develop fully automated solutions that leverage 
novel technologies to create innovative services delivered through forklift trucks. 
The vehicles are developed while keeping safety and high-performance as the main priorities.  

iGo neo shown in Figure \ref{iGoNeo} is one of the main products developed by the department, it is a low level order picker transformed 
into the agent's autonomous assistant. Functioning in autonomous or semi-autonomous modes, it can follow 
the operator and their pace while avoiding obstacles and perceiving their surroundings as well as pick 
and place pallets in designed areas. Its added value is in preserving ergonomics of the operators by 
preventing heavy load carrying for long distances and decreasing the driving ascents and descents by 75\% 
thus increasing the personal and collective performances \cite{R3}.

\begin{figure}[H]
    \begin{center}
     % Requires \usepackage{graphicx}
    \includegraphics[width=7in]{images/Chap0/KION_Hierarchy.png}\\
    \caption{KION Executive Board responsibilities as of 01.2024 \cite{R2}}
    \label{KION Hierarchy}
    \end{center}
    \end{figure}

As STILL specializes in forklift trucks, it counts many other products. Trucks are either Diesel or 
Gas fueled, or electric trucks that use Li-Ion batteries. Depending on the client's warehouse type, they can
choose from a vast range of reach trucks Figure \ref{Reach trucks}, hand pallet trucks Figure \Ref{hand truck}, 
double stacker trucks Figure \Ref{double-}, and Automated industrial Trucks Figure \Ref{iGoNeo} \cite{R4}.


\begin{figure}[h!]
    \centering
    \begin{minipage}{0.45\textwidth}
        \centering
        \includegraphics[width=\linewidth]{images/Chap0/Reach trucks.png} % Replace with your figure
        \caption{STILL reach truck}
        \label{Reach trucks}
    \end{minipage}
    \begin{minipage}{0.45\textwidth}
        \centering
        \includegraphics[width=\linewidth]{images/Chap0/hand truck.png} % Replace with your figure
        \caption{STILL hand truck}
        \label{hand truck}
    \end{minipage}
\end{figure}

\begin{figure}[H]
    \centering
    \begin{minipage}{0.45\textwidth}
        \centering
        \includegraphics[width=\linewidth]{images/Chap0/double-.png} % Replace with your figure
        \caption{STILL reach truck}
        \label{double-}
    \end{minipage}
    \begin{minipage}{0.45\textwidth}
        \centering
        \includegraphics[width=\linewidth]{images/Chap0/iGoNeo.jpg} % Replace with your figure
        \caption{STILL hand truck}
        \label{iGoNeo}
    \end{minipage}
\end{figure}

Despite the impressive capabilities of the iGo neo and similar autonomous vehicles, the implementation 
of such advanced technology brings up several challenges, particularly in ensuring reliable and predictable 
behavior under all operating conditions. This leads to a key motivation for further investigation and improvement 
in the field.

\section{Graduation project Motivation and Specifications}

\subsubsection{Motivation and Problem statement}

While autonomous vehicles can be highly reliable and efficient in carrying out various 
tasks, their behavior is not always predictable or easily explained. The output often 
exhibits a stochastic nature. For example, an obstacle-avoiding solution planned by 
the autonomous vehicle may be safe and correct but might follow an unusually shaped path. 

Such stochastic behaviors can lead to a lack of trust and interest in robotized forklift 
trucks from a customer’s perspective. This unpredictability can cause customers to 
question the system's repeatability, fearing that it may not perform consistently in 
critical situations. Moreover, the unexpected nature of these behaviors can make it 
difficult for operators to understand and anticipate the vehicle's actions, further 
reducing confidence. 

Adding to these concerns, many autonomous systems, particularly in the intralogistics 
sector, require significant commissioning efforts before they can be implemented in 
a new environment and begin their service. Whether it's a required software, sensors, or 
measurements, these systems demand substantial time and financial investment—two 
crucial resources that we aim to optimize. 

As engineers, we are committed to developing optimal solutions that are easy to 
commission in a new environment. These so-called "plug-and-play" solutions reduce 
the effort required and allow customers to start benefiting from the autonomous 
features with just the physical truck on-site and some basic input from the 
warehouse map, the rest, is online recognition and processing. This approach 
significantly enhances the impact and convenience of the technology. 

To address these issues, the autonomous vehicles department is dedicated to creating 
solutions that are not only reliable and efficient but also transparent and 
understandable. By focusing on explainable autonomous systems, the department aims 
to build greater trust with customers, ensuring they feel confident in the technology 
and are more likely to adopt and utilize these advanced robotic solutions. 

This thesis discusses one such possible application of autonomous vehicles. The use 
case involves solving the following problem: as illustrated on Figure \Ref{docked station}, 
after the vehicle enters the station- a limited area inside the warehouse where the shelf stands 
to palletize/depalletize, in predefined positions, it faces the following problematics:  

\begin{itemize}
    \item The vehicle’s forks are not facing the destination shelf but rather the opposite direction, 
    so a driving direction change is needed. 

    \item The vehicle is heavy (1200 to 1500 KG) with an overall length of 2500 to 4000 mm  
    which makes it both challenging and dangerous to change directions: turning on the spot or 
    navigating in highly curved paths\cite{R5}. 
    
    \item The pallet docking process has to be very precise to avoid shifts and mistakes. 
\end{itemize}

\begin{figure}[H]
    \begin{center}
       \includegraphics[width=5in]{images/Chap0/station-sketch.drawio.png}\\
       \caption{Vehicle setting in the station}
       \label{docked station}
       \end{center}
\end{figure}


The first inspiration for the proposed solution was the forklift drivers themselves. The experienced 
drivers all agree to solve the problematic – if it was to be solved manually, in the same way: to drive 
in an arc shape to a point, then to change the driving direction and orienting the vehicle to the 
destination position. 

We would like to solve this problematic in a manner that: 

\begin{itemize}
    \item Imitates the manual driving process to pick a pallet when in the same situation (facing backward 
    of the pallet).
    \item Implements a predictable local path planning algorithm for the vehicle operations inside a station. 
    \item Reduce the computational expense used with the global dynamic path planning algorithms implemented 
    through a pattern 
    \item Chooses an optimal path out of the various paths that can be driven.

\end{itemize}
\subsection{Project specifications}
The aim of this thesis is to devise and assess a methodology for automated path planning of a mobile 
robot, taking into account environmental obstacles. The proposed path should optimally connect the 
robot to a nearby target pose, whereby optimality is defined as achieving maximum speed while adhering 
to the robot’s kinematic constraints. Moreover, the path planning process should employ an online 
pattern-based approach, ensuring that the robot’s behavior is explainable, and that all environmental 
recognition information is utilized during path planning. 
The work should build on existing work in the Robotics Application Construction Kit (RACK) and satisfy 
real-time processing capabilities. Currently, existing methods in the literature shall be taken into 
consideration so that the functionality is sufficient for typical intralogistics applications and 
available computing power on the mobile robot itself. 

The thesis consists of the following steps: 
\begin{itemize}
    \item 	getting acquainted with the current topic area with subsequent clustering and discrimination 
    of available approaches based on a scientific literature review.
    \item  	design methods for evolving B-Spline based path planning approaches under consideration of 
    typical patterns used by manual driven trucks in intralogistics.
    \item   design methods for determining optimization approaches and related metrics to enable 
    real-time capable path planning.
    \item  	implement the derived approaches in the RACK framework and evaluate the performance of 
    the developed methods. 
    \item	verify and evaluate the developed robotic application on a mobile robot under consideration 
    of intralogistics boundary conditions.
\end{itemize}

\section{Work Structure and Methodology}

Our team adopts an Agile Scrum methodology showcased in Figure \Ref{Agile Scrum Process} to ensure 
efficient and flexible project management. Here’s how we approach our work:

\begin{figure}[H]
    \begin{center}
       \includegraphics[width=6in]{images/Chap0/blog-scrum-process-opt.jpg}\\
       \caption{Agile Scrum Process \cite{R6}}
       \label{Agile Scrum Process}
       \end{center}
\end{figure}

\subsection{Agile Scrum Framework} 
\begin{itemize}
    \item \textbf{Jira: } We use Jira to organize and track our tasks and progress. Jira allows us to create 
    and manage tickets, which are detailed records of tasks, bugs, or features that need attention. 
    Each ticket is assigned to team members and tracked through its development stages until completion.

    \item \textbf{Sprints: } Our work is organized into 2-week sprints. Each sprint is a focused period 
    where we aim to complete a set of predefined tasks. At the start of each sprint, we hold a meeting to 
    review the previous sprint: every team member presents their completed tickets, and communicates the 
    changes or blockers that appeared during the process and plan for the next sprint: decide which tasks 
    will be tackled during the sprint. 
    This helps us maintain a steady pace and regularly deliver increments of our project.

    \item  \textbf{PI Planning: } Every quarter, we engage in Program Increment (PI) planning with the 
    mobile automation teams. The PI happens in two phases: each team prepares their planning for the next 3
    months, then it is discussed and tailored again in a bigger round. This planning session helps us align 
    our goals and strategies for the upcoming quarter. We review progress, set objectives, and coordinate 
    with other teams to ensure that our work is aligned with broader project goals and company vision.

    \item \textbf{Daily Standups: } We hold 15 minutes long daily standup meetings to keep everyone on the 
    same page. During these meetings, team members share updates on their progress, discuss any challenges 
    they are facing, and outline their plans for the day. This practice promotes transparency and quick 
    problem-solving.

\end{itemize}

\subsection{Version Management}
\begin{itemize}
    \item \textbf{GitHub: } We use GitHub for version control and code management. GitHub allows us to 
    collaborate on code, track changes, and manage different versions of our project. Each team member 
    can contribute to the codebase, and we use pull requests to review and integrate new features.
\end{itemize}

\subsection{Communication and Collaboration:}

\begin{itemize}
\item  \textbf{Microsoft Teams: } We use Microsoft Teams for real-time communication and collaboration. 
Teams provides a platform for chatting, video calls, and sharing files, facilitating smooth and efficient 
interactions among team members.

\item  \textbf{Microsoft Outlook: } Outlook is used for email communication and scheduling. It helps us 
manage meetings, track important messages, and coordinate tasks and deadlines.
\end{itemize}

By integrating these tools and practices, we ensure a structured yet flexible workflow, enabling us to adapt to changes, communicate effectively, and deliver high-quality results.