\chapter{Project scope}

\renewcommand{\chaptername}{Chapter}

\section*{Introduction}


\begin{sloppypar}
\section{Host company: SmartLab Solutions GmbH}
\end{sloppypar}

The following section presents SmartLab Solutions GmbH (SLS) as the host company as well as its products.

\subsection{General information about SLS and its vision}

SLS is a German start-up based in Dresden, Saxony, and was founded in 2021. It aims to revolutionize the way laboratories manage their processes by providing a full service for laboratory digitization through analyzing, planning, and implementation.

SLS plans and builds complete solutions for laboratory digitization and offers comprehensive advice through designing solutions to assist laboratories and enhance their efficiency. \cite{R7}

The company is composed of 10 employees during the period in which this internship took place. The organizational chart of the company is detailed in Figure \ref{SLS Hierarchy}. \cite{R9}

\begin{figure}[H]
\begin{center}
 % Requires \usepackage{graphicx}
\includegraphics[width=5in]{images/Chap0/hierarchy.png}\\
\caption{SmartLab Solutions GmbH hierarchy}
\label{SLS Hierarchy}
\end{center}
\end{figure} 

\subsection{iHEX system: SmartLab Solutions GmbH main product}

Smartlab Solutions GmbH is developing digitization solutions for laboratories. The main activities are the engineering, development and production of customized solutions for customers, as well as developing a generic and modular solution which is based on smart and connected laboratory benches called \textbf{iHEX system}.

By attaching hexagon shaped elements - called \textbf{iHEX elements}, together, the needed lab devices held and controlled by the iHEX element, will be connected automatically together and everything is close-by, which enormously enhances efficiency since people do not have to make many footsteps to reach devices.

As shown in Figure \ref{iHEX photo} below, connected iHEX elements form \textbf{an island}. The island could be composed of:
\begin{itemize}
    \item Static iHEX elements: These are the foundational components of the iHEX ecosystem, designed to serve as stationary nodes within a laboratory setup. They provide a stable and reliable base for various experiments and processes.
    \item Mobile iHEX elements: In contrast, mobile iHEX elements introduce a dynamic dimension to the system. They are specially designed components that can be moved and connected to static elements. These mobile units extend the versatility of the iHEX system by facilitating flexible experimentation setups and accommodating changes in laboratory configurations.
\end{itemize}

The iHEX system solution focuses mainly on ensuring:
\begin{itemize}
    \item Seamless integration from sample to results: Thanks to connectivity, it assists the user with the data analytic. It also ensures flexibility of the workflow depending on the user needs.
    \item Workflow automation beyond individual process: Thanks to the fully integration of the process, the workflow is automated and easily controlled.
\end{itemize}

Financially speaking, the iHEX solution offers cost-efficiency since the customer can start with few modules, and then upgrade if necessary. \cite{R8}

\begin{figure}[H]
\begin{center}
 % Requires \usepackage{graphicx}
\includegraphics[width=6.5in]{images/Chap0/iHEX System.jpg}\\
\caption{SLS GmbH - iHEX product}
\label{iHEX photo}
\end{center}
\end{figure} 


The iHEX system is basically a geometry solution with a high degree of digitization. It allows the user to interface with the different laboratory components, give them orders and receive feedback:
\begin{itemize}
    \item LED strip control: Each iHEX element is equipped by an LED strip to give the user information about each device and element states.
    \item Buzzer control: As an additional informative signal, the iHEX system offers the feature of auditorily signalling the user.
    \item Power switching: Each iHEX element is capable of controlling the electricity to power on or off the lab device.
    \item Reporting the connection of a new iHEX mobile element: As a feedback, the iHEX system offers the functionality of reporting such events.    
\end{itemize}

The iHEX system assisted by a Berlin-based company product, offers a dash-boarding service (Dashboard - GUI (Graphical User Interface) in Figure \ref{iHEX ooverview} in the following page) in order to monitor all the data flow inside the laboratory.

Each iHEX element - modeled in a blue hexagon in Figure \ref{iHEX ooverview}, is equipped with a Sub-Controller (SC). All the SCs communicate with a central unit called Main Controller (MC) (Green rectangle in Figure \ref{iHEX ooverview}). The MC communicates with the server which is its intermediate with the Dashboard.

The GUI is able to interface with the hardware part via the server using an MQTT (Message Queuing Telemetry Transpot) interface; all the commands and reports are exchanged via the server. To do so, the GUI only interacts with its one counterpart; the MC, which is responsible for controlling one island.

\begin{figure}[H]
\begin{center}
 % Requires \usepackage{graphicx}
\includegraphics[width=5.5in]{images/Chap0/product overview.jpg}\\
\caption{iHEX system overview}
\label{iHEX ooverview}
\end{center}
\end{figure} 


The high-level part of the project, shown in a green frame in Figure \ref{iHEX ooverview}, is handled by the software development team, whereas the low-level part, shown in a blue frame in Figure \ref{iHEX ooverview}, is handled by the engineering team to which I belong.


\section{Project specifications}

This project consists of developing, implementing, and testing a low-level communication protocol firmware for the iHEX system to ensure the control of the bench elements.

The following are the different specifications required by the company:

\begin{itemize}
    \item \textbf{Choose the appropriate communication protocol and the different platforms:}
    This part consists of analyzing the available communication protocols in the market, detailing their characteristics, and choosing the most appropriate one based on the different project constraints.
    
    After making the decision, it is important to choose the platforms to work on within this project. Choosing these platforms implies doing the technico-economical study and decide the platforms based on different criteria.

    \item \textbf{Ensure the communication between the elements}:
    This part consists of developing the needed firmware for the chosen platforms. These firmware provide the needed APIs (Application Programming Interfaces)  for the application development.

    \item \textbf{Control Interface (CI) development}:
    Based on the functional specifications provided by the company, we have to elaborate the operational specifications to ensure the needed functionalities and develop the application software for all the chosen platforms. The functionalities are communication, LED control, powering control, devices control, feedback reception, (etc.).

    \item \textbf{Hardware development}:
    After the prototype validation (software and hardware parts), it is essential to proceed to finalize the hardware development. It consists of designing and developing the necessary electronic gadgets and the Printed Circuit Boards (PCBs) that assemble all the electronic modules to ensure the functionalities required by the company.
\end{itemize}

\section{Project management tools}
\hyphenation{manage-ment}
To ensure the successful delivery of embedded software projects, effective project management approaches are necessary.

This part consists of introducing the work methodology, the managerial strategy, and the project management tools used throughout the project.

\subsection{Agile methodolgy}

We adopted the Agile methodology for the whole project as it has shown its ability to improve efficiency.

This method consists of dividing the project into sprints, each sprint is also divided into cards (elementary tasks). Using this methodology, we aim to increase transparency, flexibility, and enhancement. \cite{R10}

Once all the sprints and cards are created and scheduled, the work starts respecting the following rules:
\begin{itemize}
    \item \textbf{Bi-weekly meetings}:  Evaluate the sprint and check for global advancement.
    \item \textbf{Frequent interactions between team members}: Good communication is essential to make this step successful.
\end{itemize}

All the Agile methodology phases are indicated in Figure \ref{Agile}.

\begin{figure}[H]
\begin{center}
 % Requires \usepackage{graphicx}
\includegraphics[width=5in]{images/Chap0/Agile.png}\\
\caption{Agile methodology}
\label{Agile}
\end{center}
\end{figure} 

\subsection{Project management software}

In order to assist the management job, several software tools are available.

\subsubsection{Microsoft tools: Microsoft Teams - Microsoft Outlook}

Microsoft Teams is a powerful project management tool that provides a range of features to support the needs of Agile teams. With Teams, team members can communicate in real-time, share files and documents, and collaborate on tasks and projects.

As for Microsoft Outlook, it is a widely-used email client that provides a range of features to support communication and collaboration, including email, calendar, and task management.

\subsubsection{YouTrack}

YouTrack is a comprehensive project management tool that is designed specifically for Agile teams. The platform provides a range of features to support Agile methodologies:

%\begin{itemize}[label=\ding{70}]

\begin{itemize}
    \item \textbf{Tickets (cards) tracking}: Tickets are used to assign micro tasks to the Agile team's members and track their signs of progress.
    Each ticket can have only one status at a time. Depending on its activity and workflow, each company chooses the appropriate set of statuses.
    
    \item \textbf{sprints tracking}: YouTrack offers the feature of recapitulating the different tickets of the sprint.
    \hyphenation{expla-nation}
    \item \textbf{Backlog management}: The backlog contains all the awaiting tasks with their explanation.

    \item \textbf{Dashboarding}: YouTrack dashboards are a powerful feature that makes the team members focused on their goals. Customizable dashboards are so advantageous as they allow the members to select the information that matters most to them.
    This feature provides real-time visibility which enables team members to quickly identify issues and take action as needed.
\end{itemize}

YouTrack offers also the Kanban board which is a lean method used in agile project management.
SLS GmbH Kanban board is shown in Figure \ref{Kanban}   .

\begin{figure}[H]
\begin{center}
 % Requires \usepackage{graphicx}
\includegraphics[width=5in]{images/Chap0/SLS KANBAN board.jpg}\\
\caption{SLS GmbH Kanban board}
\label{Kanban}
\end{center}
\end{figure} 

\subsubsection{GitHub}
GitHub is a popular software development platform that is widely used by Agile teams to manage code repositories and collaborate on projects. 

The platform provides a range of features to support project management, including issue tracking and pull requests.
\begin{itemize}
    \item \textbf{Issue tracking}: With this feature, team members can easily report and track bugs and other issues, assign tasks to team members, and monitor progress.
    
    \item \textbf{Pull request}: This feature allows team members to propose changes to code and collaborate on code reviews before merging changes into the codebase.
\end{itemize}

\section*{Conclusion}
Chapter 1 has presented the host company and its vision. Then, it detailed the general context and the objectives of this project, as well as the adopted working methodology. In the subsequent chapter, we will introduce the different milestones we went through to select the appropriate communication protocol and the MC and SC development boards. Then, we will describe the different steps to develop the communication software on both the MC and SC.