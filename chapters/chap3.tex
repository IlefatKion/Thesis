\chapter{Design and development of Printed Circuit Boards}

\renewcommand{\chaptername}{Chapter}

\section*{Introduction}

In Chapter 4, we dive into the essential process of creating the electronic foundation of the iHEX system—Printed Circuit Boards (PCBs). These PCBs serve as the crucial framework where all the electronic parts of the iHEX system come together, ensuring seamless communication and control. This chapter explores the design, development, and roles of four specific PCB types: Main PCB, IO PCB, LED PCB, and Dock PCB.

\section{Architecture}

Based on the specifications and the company needs, we designed the global hardware architecture. It is mainly composed of 4 designed PCBs:

\begin{itemize}
    \item Main PCB is the motherboard that brings the main electronic components of the SC together. Each SC is composed of one Main PCB that is connected directly to an IO PCB.
    \item IO (Input-Output) PCB mainly ensures connectivity and interaction with the other SCs. It is connected to the Main PCB on one side. On the other side, it is connected whether to IO PCBs of other static SCs or Dock PCBs.
    \item Dock PCB ensures the interaction and control of the iHEX mobile element newly connected. If it exists, whether in a static or mobile element, it is connected to the SC through the IO PCB.
    \item LED PCB controls the LED strip. Each LED PCB is part of a SC. It is plugged-in directly on the Main PCB.
\end{itemize}

The following Figure \ref{connection PCB} explains graphically the different connections between the PCBs within the suggested architecture:

\begin{figure}[H]
\begin{center}
\includegraphics[width=5in]{images/Chap3/PCB architecture.png}\\
\caption{The different connections between the PCBs}
\label{connection PCB}
\end{center}
\end{figure}

In Figure \ref{connection PCB} \texttt{SC1}, \texttt{SC2}, and \texttt{SCn} are static elements. They are part of the same island. They are connected to each other thanks to their IO PCB. \texttt{SCn+1} is a mobile element.

\texttt{SC1} is equipped by a Dock that offers the possibility of connecting a new mobile element. The Male Dock PCB is connected to \texttt{SC1} via its IO PCB. Same for the Female Dock PCB, it is connected to SCn+1 via its IO PCB.

To design the needed PCBs, we use the Computer Aidied Design (CAD) software KiCad \cite{R29}.

\section{Main PCB: Nucleus of the iHEX hardware}

As the central part of the SC, the Main PCB shown in Figure \ref{Main PCB overview}, basically assembles the different electronic components that define its functionality. This section details the design, integration, and functionality of the Main PCB—where the NUCLEO-F303RE microcontroller, CAN transceiver, power supply, LED controllers, and more converge to drive the iHEX system's SC capabilities.

\begin{figure}[H]
\begin{center}
\includegraphics[width=5in]{images/Chap3/Main pcb final.png}\\
\caption{Main PCB overview}
\label{Main PCB overview}
\end{center}
\end{figure}

\subsection{Electronic components}

The following are the components that are part of the Main PCB:

\begin{itemize}
    \item \textbf{NUCLEO-F303RE MCU:} The NUCLEO-F303RE microcontroller serves as the system's brain, responsible for processing instructions and coordinating actions across the entire iHEX network. Its integration was facilitated by a meticulously downloaded package from SnapEDA, ensuring precise footprints and compatibility.
    \item \textbf{CAN transceiver:} The Main PCB integrates a custom-designed CAN transceiver footprint, enabling smooth communication between various elements of the iHEX system. This CAN transceiver bridges the gap between the CAN bus network and the microcontroller, allowing for reliable data transmission and reception.
    \item \textbf{Header connectors:}
    A matrix of header connectors emerges as the physical interface between the Main PCB and the iHEX system's external elements. The needed headers are summarized in Table \ref{Main PCB LED indicator} below:
    
    \begin{table} [H]
    \centering
    \begin{tabularx}{\textwidth}{|c|Y|Y|} \hline
     \textbf{Quantity} & \textbf{Component} & \textbf{Role} \\ [0.5ex] % RPi
     \hline\hline
     3 & 2-Position headers (12V-GND) & Facilitate power supply connections, enabling the efficient distribution of 12V power and ground signals. \\ 
     \hline
     2 & 4-Position headers (LED Control) & Govern the RGB LED strips' behavior, providing ground and 12V connections for dynamic illumination. \\
     \hline
     1 & 2-Position header (External Buzzer Control) & Enable external control over the system's auditory feedback via a simple signal. \\
     \hline
     1 & 2-Position header (Relay Control) & Regulate electricity to the device atop the iHEX element, enhancing control capabilities. \\
     \hline
     2 & 6-Position female headers (LED driver control) & \multirow{2}{\linewidth}{\multirowsetup{\centering Manage LED driver control signals, harmonizing the visual experience.}} \\
     \cline{1-2}
     2 & 3-Position female header (LED driver control) &  \\
     \hline
     1 & 6-Position female header (CAN transceiver) & Establish a seamless connection between the CAN transceiver and the network. \\
     \hline
     1 & 14-Position connector (Flat Ribbon Cable) & Link the Main PCB with the IO PCB, ensuring CAN communication, and signal loop and relay control signal transmission. \\
     \hline
     2 & 1-Position female header (Internal Buzzer control) & Enable internal control over the system's auditory feedback. \\
     \hline
     1 & 3-Position female header (MOSFET) & Facilitate control over the MOSFET that governs electricity supply to the system's top device. \\
     \hline
     1 & 2-Position male header (CANH and CANL for Troubleshooting and Visualization) & Provide direct access for troubleshooting and visualization of CAN signals. \\
     \hline
     1 & 2-Position male header (VCC and Ground for undefined purposes) & Offer flexible connection points for future expansion. \\
     \hline     
    \end{tabularx}
    \caption{Electronic components in the Main PCB}
    \label{electronic components Main PCB}
    \end{table}
    \newpage
    \item \textbf{Indicator LEDs:} 
    The two indicator LEDs in Table \ref{Main PCB LED indicator} contribute to the Main PCB's intuitive user interface:
    %\newcolumntype{Y}{>{\centering\arraybackslash}X}
    \begin{table} [H]
    \centering
    \begin{tabularx}{\textwidth}{|c|Y|Y|} \hline
     \textbf{Quantity} & \textbf{Component} & \textbf{Role} \\ [0.5ex] % RPi
     \hline\hline
    1 & Green LED (VCC - 3V3 Indicator) & Illuminates to signify the presence of a 3.3V supply. \\
    \hline
    1 & Yellow LED (VDD - 12V Indicator) & Lights up to indicate the availability of a 12V supply. \\
    \hline
    \end{tabularx}
    \caption{Main PCB LED indicators}
    \label{Main PCB LED indicator}
    \end{table}
\end{itemize}

\subsection{Schematic, integration, and PCB design}

The Main PCB's construction encompassed the schematic design, footprint assignment, and PCB layout. The schematic diagram, as shown in Figure \ref{Main PCB schematic}, interwove necessary wiring, outlining the connections between components. Each component's footprint was assigned to ensure a perfect fit and proper connectivity. The resulting PCB design in Figure \ref{Main PCB design}, encapsulates the harmonious fusion of components, underscoring the NUCLEO-F303RE's principal role of controling the iHEX system.

\begin{figure}[H]
\begin{center}
\includegraphics[width=3in]{images/Chap3/Main pcb.png}\\
\caption{Design of the Main PCB}
\label{Main PCB design}
\end{center}
\end{figure}

\section{IO PCB: Connectivity and control}

The IO PCB, as shown in Figure \ref{IO PCB overview}, serves as the interface that empowers the connected SC to connect, communicate, and control the other members of the island (MC and other SCs).

\begin{figure}[H]
\begin{center}
\includegraphics[width=5in]{images/Chap3/IO PCB final.png}\\
\caption{IO PCB overview}
\label{IO PCB overview}
\end{center}
\end{figure}

\subsection{Ethernet ports:}
At the main part of the IO PCB the Ethernet ports play a distinctive role in facilitating communication and connectivity:

\begin{itemize}
    \item 2 Input Ethernet Ports: Establish a vital link between the current iHEX element and the CAN bus via CANH and CANL signals. These ports serve for communication between the SC and the other parts of the island (MC and other SCs).
    \item 5 Output Ethernet Ports: These ports accommodate both static and mobile iHEX elements' unique needs:
    \begin{itemize}
        \item Static iHEX Element Output: For static elements, these ports offer CANH and CANL terminals to facilitate the connection of subsequent static iHEX elements to the CAN bus.
        \item Mobile iHEX Element Output (Connected to Dock): When connected to a mobile iHEX element via the Dock, these ports become the interface of connectivity and control:
        \begin{itemize}
            \item CANH and CANL Terminals: Transmit CAN communication signals, allowing integration into the iHEX network.
            \item Signal Loop Detection: Receive the signal loop value (3.3V or GND) from the dock, detecting connection or disconnection of mobile elements.
            \item 12V Power Supply: Provide a regulated 12V power supply to the relay for electricity control.
            \item Relay Trigger: Receive a 3.3V signal to trigger the relay, enabling precise control over electricity supply.
        \end{itemize}
    \end{itemize}
\end{itemize}

\subsection{Header connectors}

The IO PCB features essential header connectors (Table \ref{IO PCB headers}) that facilitate communication and integration with the Main PCB and other system elements:

    %\newcolumntype{Y}{>{\centering\arraybackslash}X}
    \begin{table} [H]
    \centering
    \begin{tabularx}{\textwidth}{|c|Y|Y|} \hline
     \textbf{Quantity} & \textbf{Component} & \textbf{Role} \\ [0.5ex] % RPi
     \hline\hline
     1 & 14-Position Header (Flat Ribbon Cable) & Form a vital connection to the Main PCB, transmitting the signal loop information, receiving relay control signals, and facilitating CAN bus connectivity via CANH and CANL terminals. \\
     \hline
     1 & 2-Position Header (12V Power Supply) & Provide a standardized 12V power supply to system elements, ensuring consistent and reliable operation. \\
     \hline
    \end{tabularx}
    \caption{IO PCB Header}
    \label{IO PCB headers}
    \end{table}


\subsection{Indicator LEDs}
Two indicator LEDs (Table \ref{IO PCB LED indicator}) provide instant visual feedback on the IO PCB's status:
    %\newcolumntype{Y}{>{\centering\arraybackslash}X}
    \begin{table} [H]
    \centering
    \begin{tabularx}{\textwidth}{|c|Y|Y|} \hline
     \textbf{Quantity} & \textbf{Component} & \textbf{Role} \\ [0.5ex] % RPi
     \hline\hline
     1 & Green LED (3.3V Indicator) & Lights up to signal the presence of a 3.3V supply, confirming the availability of essential power. \\
     \hline
     1 & Yellow LED (12V Indicator) & Provide a standardized 12V power supply to system elements, ensuring consistent and reliable operation. \\
     \hline
    \end{tabularx}
    \caption{IO PCB LED indicators}
    \label{IO PCB LED indicator}
    \end{table}

\subsection{Schematic integration, and PCB design}
We brought the IO PCB to life through creating a schematic design (Figure \ref{IO PCB schematic}), assigning footprint, and developing the PCB layout (Figure \ref{IO PCB design}). The schematic diagram outlines wiring connections and interactions between components, ensuring the different functionalities. The assignment of each component's footprint guarantees proper fitment and effective integration. The PCB in Figure \ref{IO PCB design} presents the final result.

\begin{figure}[H]
\begin{center}
\includegraphics[width=5in]{images/Chap3/IO PCB design.png}\\
\caption{Design of the IO PCB}
\label{IO PCB design}
\end{center}
\end{figure}

\section{Dock PCB: Connecting and powering iHEX mobile elements}

Dedicated to the task of connecting mobile elements to static ones, the Dock PCB (Figure \ref{Dock PCB overview}) stands as a bridge that enables the integration between these elements. This section details the design and integration of the Dock PCB, taking in consideration the Male Dock and Female Dock compositions.

\begin{figure}[H]
\begin{center}
\includegraphics[width=3in]{images/Chap3/Dock PCB image.png}\\
\caption{Dock PCB overview}
\label{Dock PCB overview}
\end{center}
\end{figure}

\subsection{Male Dock}
The Male Dock presents the PCB that is integrated in the Dock unit if it exists. In junction with the Female Dock, it contributes to the integration of the iHEX mobile elements. It is mainly composed of:

\begin{itemize}
    \item Ethernet port: It is used for connecting the Dock to the SC via the IO PCB Output port. It contributes to the establishment of a bidirectional communication between mobile element and the other elements of the island.
    \item MOSFET: It triggers the relay that governs the power supply of the mobile element.
    \item Green LED (Relay Control Visualizer): It offers instant visual feedback, illuminating when the relay control signal is active, and extinguishing when inactive. This LED enables visual debugging.
    \item 2 * 3-Position headers: These connectors provide essential connectivity:
    \begin{itemize}
        \item CANH and CANL Terminals: Establish CAN communication channels, ensuring fluid integration into the iHEX network.
        \item Signal Loop detection: Receive the signal loop information, indicating the connection status of a new mobile element.
        \item Relay control signal: Enable the power supply activation and deactivation of mobile elements.
        \item Common Ground: Ensure common ground for all the iHEX system.
    \end{itemize}   
\end{itemize}

\subsection{Female Dock}
The Female Dock presents the PCB that is integrated in the Dock unit if it exists. In junction with the Male Dock, it contributes to the integration of the iHEX mobile elements. It is mainly composed of:

\begin{itemize}
    \item Ethernet port: It establishes a direct connection to the IO PCB's Input Ethernet port.
    \item 1 * 3-Position Header: This connector forms the foundation of essential connections:
    \begin{itemize}
        \item CANH and CANL terminals: Create a vital link for CAN communication, enabling data exchange with the iHEX network.
        \item Common Ground: Ensure common ground for all the iHEX system.
    \end{itemize}    
\end{itemize}

\subsection{Schematic, integration, and PCB Design}

We brought the Dock PCB to life through creating a schematic design (Figure \ref{Dock PCB schematic}), assigning footprint, and developing the PCB layout (Figure \ref{Dock PCB design}). The PCB in Figure \ref{Dock PCB design} presents the final result.

\begin{figure}[H]
\begin{center}
\includegraphics[width=2.5in]{images/Chap3/Dock PCB design.png}\\
\caption{Design of the Dock PCB}
\label{Dock PCB design}
\end{center}
\end{figure}

\section{LED PCB: LED strips control}

Dedicated to visual signalling the user, the LED PCB (Figure \ref{LED PCB overview}) controlled by the SC MCU, controls the LED strips of the  iHEX element. This section unveils the design and integration of the LED PCB.

\begin{figure}[H]
\begin{center}
\includegraphics[width=2.2in]{images/Chap3/LED PCB.png}\\
\caption{LED PCB overview}
\label{LED PCB overview}
\end{center}
\end{figure}

\subsection{RGB color control}

The LED PCB is mainly composed of four sophisticated MOSFETs:

\begin{itemize}
    \item 3 MOSFETs for RGB color control: These dynamic components govern the intensity of each primary color—Red, Green, and Blue. That is achieved by modulating the voltage across the LEDs using PWM signals.
    \item 1 MOSFET for blinking control: A singular MOSFET assumes the role of controlling the blinking frequency and duty cycle of the control PWM signal.
\end{itemize}

 \subsection{Input and Output connections}

The LED PCB takes LV (Low-Voltage) control signal (3v3 - GND) as input and delivers MV (Medium-Voltage) control signal (12v - GND). Two connectors make the LED PCB interact with the external electronic components:

 \begin{itemize}
     \item 1 * 6-Position input connector: This input hub encapsulates the RGB color and blinking control:
    \begin{itemize}
        \item R, G, B, Blk (Blinking) PWM inputs: CMOS PWM signals that control the the colors and the blinking frequency and duty cycle.
        \item 12V and GND inputs: Provide the necessary power to illuminate the LEDs.
    \end{itemize}  
    \item 1 * 3-Position output connector: MV PWM signals. It is the amplified image of the CMOS control signal received by the LED PCB.
 \end{itemize}

\subsection{Schematic integration, and PCB design}

The LED PCB's design process starts with a crafted schematic diagram (Figure \ref{LED control schema}). To each electronic component and depending on the used reference, we assign the corresponding footprint. Finally, we design the PCB (Figure \ref{LED PCB design}) which is the final step before printing the final board.

\begin{figure}[H]
\begin{center}
\includegraphics[width=3in]{images/Chap3/LED controller PCB design.png}\\
\caption{Design of LED PCB}
\label{LED PCB design}
\end{center}
\end{figure}

%TODO: photos of the final sub-controllers

\section{Production, test and validation}

The design and development efforts found realization through the production of the PCBs. The production phase was entrusted to "Pcbcart" \cite{R32}, a leading Chinese company renowned for its precision and quality. To do so, we generated the necessary Gerbers (.gbr) and Drill Files (.drl) for the fabrication: routes and drills.  Upon receiving the fabricated boards, the electronic components which are delivered by trusted suppliers such as Digikey \cite{R33}, Mouser \cite{R34}, and Würth Electronik \cite{R35} are soldered on the PCBs.

Assembling PCBs, electronic elements, and electrical components lead to obtaining the final SC (Figure \ref{SC wiring}) and Dock boxes, laying the groundwork for comprehensive testing and validation. 

\begin{figure}[H]
\begin{center}
\includegraphics[width=4in]{images/Chap3/SC wiring.jpg}\\
\caption{SC wiring}
\label{SC wiring}
\end{center}
\end{figure}

In the company's showroom, the island is mainly composed of six iHEX elements and one mobike element equipped each of a SC (Figure \ref{SC final}). Two static elements are equipped each with a Dock (Figure \ref{Dock final}) allowing the connection of the mobile element.

\begin{figure}[H]
\begin{center}
\includegraphics[width=4in]{images/Chap3/SC final.jpg}\\
\caption{SC final box}
\label{SC final}
\end{center}
\end{figure}

\begin{figure}[H]
\begin{center}
\includegraphics[width=4in]{images/Chap3/Dock final.jpg}\\
\caption{Dock final box}
\label{Dock final}
\end{center}
\end{figure}

The finalized PCBs and the developed CI software are ready to be tested. We make the necessary wiring, bring everything together and start the test and validation process. During this phase, we test the communication across both channels and the iHEX elements ability to correctly respond to the MQTT commands, notably the LED strips and Buzzers control as well as powering on and off the electrical devices. We also have to validate the mobile element connection on one Dock through the reception of a significant MQTT message.
 
Succeeding the testing phase makes us not only validate our engineering job of design and development of the PCBs, but also make sure of the integrity of the software and hardware final product.

\section*{Conclusion}

Chapter 4 took us through the process of making important PCBs. We designed these boards, and then turned the designs into real boards with the help of a partner company. After putting the parts on the boards and connecting everything, we tested them to make sure they work well and adapted to the software in real situations. This process shows how software and hardware come together to make the iHEX system work smoothly.
