%---------------------------------------environment abstract
\newenvironment*{Abstract}{
\renewcommand*{\abstractname}{\Huge \textbf{Abstract}}
\begin{abstract}
\end{abstract}}


%---------------------------------------environment dedication
\newenvironment*{dedication}{
\renewcommand*{\abstractname}{\begin{flushleft}\Huge \textbf{Dedication}\end{flushleft}}
\begin{abstract}
\end{abstract}}

%---------------------------------------environment acknoledgement
\newenvironment*{acknowledgement}{
  \renewcommand*{\abstractname}{\begin{flushleft}\Huge \textbf{Acknowledgement}\end{flushleft}}
  \begin{abstract}
 {}
  \end{abstract}
}
%==========================================================================


%----------------------------------------- Acknowledgement
\newpage

\begin{dedication}

\end{dedication}

%---------------------------------------------Acknowledgement
\begin{acknowledgement}

\end{acknowledgement}

%---------------------------------------------Abstract
\begin{abstract}

      This thesis proposes a methodology for automated path planning for mobile robots in intralogistics 
      environments, taking into consideration environmental obstacles. The proposed path should optimally 
      connect the robot to a nearby target pose, while adhering to the robot’s kinematic constraints. The 
      path planning process should employ an online pattern-based approach, ensuring that the robot’s 
      behaviour is explainable, and all environmental recognition information is utilized during path planning. 
      The thesis aims to reduce the amount of information required for commissioning mobile robots, allowing for 
      broader use cases. The proposed solution will be implemented in the Robotics Application Construction Kit 
      (RACK) framework and evaluated for performance in real-time processing capabilities. The thesis consists 
      of literature review, design and development of the proposed solution, and validation through testing in 
      a simple test environment. The proposed methodology will enable mobile robots to navigate intralogistics 
      environments more efficiently and safely, contributing to increased productivity and dynamic logistics 
      processes.



    
    \textbf{Keywords: Near-Field Path Planning, Optimization Algorithms, Intralogistics, AMR }

    

    

\end{abstract}







