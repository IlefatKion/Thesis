%---------------------------------------environment abstract
\newenvironment*{Abstract}{
\renewcommand*{\abstractname}{\Huge \textbf{Abstract}}
\begin{abstract}
\end{abstract}}


%---------------------------------------environment dedication
\newenvironment*{dedication}{
\renewcommand*{\abstractname}{\begin{flushleft}\Huge \textbf{Dedication}\end{flushleft}}
\begin{abstract}
\end{abstract}}

%---------------------------------------environment acknoledgement
\newenvironment*{acknowledgement}{
  \renewcommand*{\abstractname}{\begin{flushleft}\Huge \textbf{Acknowledgement}\end{flushleft}}
  \begin{abstract}
 {}
  \end{abstract}
}
%==========================================================================


%----------------------------------------- Acknowledgement
\newpage

\begin{dedication}

I dedicate this work to everyone that stood by me during my journey.
First of all, to my dearest parents \textbf{Souheil} and \textbf{Ameni}
whose love and guidance nurtured me and paved the way for me in every step
of my life. I never underestimated the sacrifices you did for our little 
family. This achievement, as well as others, is a coronation of 
your efforts. Beautiful days are coming!

To my brother \textbf{Manef}, the source of joy and happiness in our family. 
Thank you for being who you are, for loving and caring for me in your own way.

To all my family members that encourage me and celebrate me like a kid of their own.

To \textbf{Amine}, my support system since day one. Thank you for believeing in me
and pushing me beyond the limits. Our collaborations and ambitions shall continue.
Let's keep dreaming.

To my second family, \textbf{IEEE RAS INSAT}: the mentors and the work buddies.
I learned a lot from the experiences and from every one there and in particular 
through the \textbf{EUROBOT2023} experience. 

To all those who have supported and believed in me, both near and far. Thank you.


\end{dedication}

%---------------------------------------------Acknowledgement
\begin{acknowledgement}

  I would like to express my heartfelt gratitude to all the individuals who contributed to the success of 
  this work through their invaluable feedback, support, and guidance.

  I extend my sincere thanks to Prof. \textbf{Feiza GHEZAIEL}, the jury president, as well as Prof. 
  \textbf{Sonia HAJRI}, the reviewer of this graduation project, for their insightful comments, constructive 
  feedback, and the generous time they dedicated to evaluating my work. Their contributions have greatly 
  enhanced the quality of this study and will continue to influence its future.
  
  I am deeply grateful to my mentor and supervisor, Mr. Dr. \textbf{Tino KR\"{U}GER-BASJMELEH}, for his 
  unwavering support, expert guidance, and profound knowledge throughout the course of my research. 
  His wisdom and advice played a crucial role in steering this work to success, and I have learned 
  immensely from him.
  
  My profound appreciation goes to Prof. \textbf{Afef BEN ABDELGHANI}, whose expertise and insightful 
  discussions have significantly enriched this research. Her guidance has been a constant source of 
  inspiration, not only throughout this project but since my earliest days at \textbf{INSAT}.
  
  I would also like to express my gratitude to my institute, \textbf{INSAT}, for the immense impact it 
  has had on both my personal and professional development. The experiences I gained and the people I 
  met there have shaped me as both a person and an engineer.
  
  I acknowledge with sincere thanks all my professors and teachers, whose efforts and dedication were 
  essential pillars in my academic journey, contributing profoundly to my knowledge and growth.
  
  Special thanks go to my esteemed colleagues at STILL for their continuous support and cooperation 
  throughout the different phases of this work. I would particularly like to thank \textbf{Felix HESS}, 
  \textbf{Kevin VORWERK}, \textbf{Matthias HAASE}, and \textbf{Martin KRIMM} for their valuable advice, 
  as well as \textbf{Thomas WITTMANN} and \textbf{Volker VIERECK} for generously sharing their expertise.
  
  The steadfast support and belief in my abilities from these individuals have profoundly influenced the 
  outcome of this graduation project. I am truly thankful for their presence and encouragement throughout 
  my academic and professional journey.
  

\end{acknowledgement}

%---------------------------------------------Abstract
\begin{abstract}

  This thesis presents a novel methodology for automated path planning in intralogistics environments, focusing on the efficient and safe navigation of mobile robots. The proposed approach addresses the challenges of real-time path planning, obstacle avoidance, and explainability.

  A pattern-based online path planning algorithm is developed, leveraging environmental recognition information to generate optimal paths that adhere to the robot's kinematic constraints. To enhance path quality, a combination of evaluation metrics and metaheuristic optimization techniques is employed. The most efficient optimization algorithm is selected based on empirical analysis.
  
  The primary objective is to reduce the commissioning effort for mobile robots, enabling broader applications in intralogistics. The proposed solution is implemented within the Robotics Application Construction Kit (RACK) framework and evaluated for its real-time performance and effectiveness in various intralogistics scenarios.
  
  The thesis structure includes a comprehensive literature review, the design and development of the path planning methodology, and rigorous testing and validation in a simulated and potentially real-world intralogistics environment. The expected outcomes include improved efficiency, safety, and adaptability of mobile robots in intralogistics operations.



        \textbf{Keywords: AMR, Near-Field Path Planning, Optimization Algorithms, Obstacle Avoidance, Intralogistics }


    

    

\end{abstract}







