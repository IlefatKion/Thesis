\chapter*{General conclusion}
\addcontentsline{toc}{chapter}{General conclusion}

This project has explored the complex and rapidly evolving field of intralogistics, focusing on the optimization of 
local path planning for the autonomous forklifts' AMRs produced by STILL. The technique introduced by this thesis 
concerns a specific part of the global path that allows the AMR to properly dock the target. The work was structured 
across four key chapters, each building towards a comprehensive understanding and solution to the challenges faced 
in modern industrial environments.

Chapter 1 provided a thorough introduction to the host company, STILL, detailing its structure, and contributions 
to the intralogistics industry. By outlining the project’s context, motivations, and core problematics, this chapter 
laid the groundwork for understanding the broader objectives and the specific goals of this thesis, particularly 
in optimizing path planning near pallet stations.

\noindent Chapter 2 delved into the state-of-the-art literature, offering a review of existing solutions in autonomous 
robotics and path planning. This chapter underscored the gaps in current approaches and the opportunity for 
innovative optimization strategies, especially in environments like brownfield warehouses, where challenges such 
as disorganization and uneven terrain present significant obstacles to automation.

\noindent Chapter 3 marked the technical core of the thesis, focusing on the mathematical and geometric principles behind 
the proposed solution. The exploration of splines in path planning, combined with the geometric division of 
warehouse stations into transition zones, highlighted the need for precision and flexibility in near-field 
navigation. Through a detailed discussion of path discrimination and optimization methods, this chapter 
demonstrated how smart algorithms can enable autonomous vehicles to plan and execute efficient routes 
while avoiding obstacles.

\noindent Finally, Chapter 4 presented the practical implementation of these ideas in the RACK framework and simulation 
system, detailing the tests conducted on autonomous forklifts and the results obtained. The successful trials 
showcased the efficacy of the optimized local path planning approach, proving that the developed solution is 
not only theoretically sound but also practically applicable in real-world scenarios.

In conclusion, this thesis has demonstrated that the desired optimizations can be be achieved by implementing less sophisticated path planning techniques and without the need to implement over-kill solutions like AI-based. The proposed approach contributes to the growing body of knowledge in autonomous robotics and intralogistics, offering a solution that enhances safety, efficiency, and predictability which opens the door for customers and potential customers to trust the presence of the autonomous forklifts in complex intralogistics environments with the european standards.