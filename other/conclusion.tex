\chapter*{General conclusion}
\addcontentsline{toc}{chapter}{General conclusion}

This project has explored the complex and rapidly evolving field of intralogistics, with a focus on optimizing 
local path planning for the autonomous forklift AMRs produced by STILL. The technique introduced by this 
thesis addresses a specific segment of the global path, enabling the AMR to accurately dock at its target. 
The research was structured into four key chapters, each contributing to a comprehensive understanding of the 
challenges faced in modern industrial environments and presenting innovative solutions.

First, the state-of-the-art chapter reviewed existing solutions in autonomous robotics path planning and 
optimization techniques. This section highlighted the shortcomings in current approaches and emphasized 
the need for new optimization strategies, particularly in environments like brownfield warehouses, where 
obstacles such as disorganization and uneven terrain hinder automation efforts. After analyzing these 
existing approaches and studying the necessary tools to address the problem statement, the adopted methodology 
was thoroughly designed and documented.

The technical core of the thesis followed, focusing on the mathematical and geometric principles underlying 
the proposed solution. The geometric division of warehouse stations into transition zones and the incorporation 
of splines for path creation emphasized the importance of precision and flexibility in near-field navigation. 
Through a detailed discussion of path discrimination and optimization techniques, this chapter demonstrated 
how smart algorithms enable autonomous vehicles to plan and execute efficient routes while avoiding obstacles.

Lastly, the practical implementation of these ideas was realized within the RACK framework and simulation 
system, detailing the tests conducted on autonomous forklifts and the results achieved. The successful 
trials underscored the effectiveness of the optimized local path planning approach, proving that the 
developed solution is both theoretically robust and practically viable in real-world scenarios. The use 
of metaheuristic optimization algorithms, such as DE and PSO, proved highly effective in significantly 
reducing planning time, leading to the recommendation of these algorithms for path optimization.

This project has shown that desired optimizations can be achieved through the application of less complex 
path planning techniques, without resorting to overly sophisticated solutions. The integration of explainable 
AI and a station-specific design has contributed to creating a solution that enhances safety, efficiency, 
and predictability, paving the way for customers to trust autonomous forklifts in complex intralogistics 
environments that meet European standards.

However, limitations exist when the AMR operates within transition zones. In such cases, the optimizer may 
generate curved paths that are not ideal for driving. Additionally, the transition zones created are not 
the only open areas in the station and are optimal only for the predefined test scenarios. If the truck 
is positioned elsewhere, such as next to a shelf, transitioning through these zones becomes suboptimal. 
Furthermore, when the station is cluttered with obstacles, the path planner struggles to generate feasible 
routes, necessitating the use of more advanced planners like those based on OMPL.

Looking ahead, expanding the number of transition zones is recommended. These zones could be placed in 
free areas of the station, such as the space between existing transition polygons, and adjusted based on 
the truck's location. For instance, if the truck is already positioned within a transition area, that 
zone should be excluded from the path planning process. Additionally, scans for surrounding obstacles 
should be confined to the station's interior, as there is no practical need to account for obstacles 
outside the station. This can be done by algorithmically checking for collisions only within the station. 
Care must be taken, however, as the AMR sometimes reaches areas outside the station. This refinement 
can significantly reduce planning time and further optimize the solution.

In conclusion, this project has successfully addressed critical challenges in AMR path planning within 
intralogistics, offering a practical, efficient, and scalable solution. The proposed methodology not 
only enhances operational performance but also introduces flexibility in adapting to real-world constraints. 
By further refining the transition zones and limiting obstacle scans, the approach holds promise for 
even greater optimization, reducing complexity while maintaining high standards of safety and efficiency.




