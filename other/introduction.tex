\chapter*{General Introduction}
\addcontentsline{toc}{chapter}{General Introduction} % to add the intro to the tables of contents
The latest events of the current decade have highlighted the challenges that manufacturers, suppliers, and end customers
face during fluctuations in logistics and supply chain processes. Living in a VUCA world—Volatile, Uncertain, Complex,
and Ambiguous—requires us to continuously adapt to changes and anticipate future events by preparing our developed
environments and scaling our solutions. Simultaneously, it is crucial to maintain high standards that ensure productivity,
enhance work safety, and optimize ergonomics. 


In this context, the primary objective of intralogistics is to optimize, integrate, automate, and manage internal 
logistical flows of material and information within distribution centers, warehouses, or manufacturing plants. 
This subfield focuses on increasing operational efficiency by employing new technologies, such as autonomous robots. 

Modernizing industrial environments through intralogistics offers significant potential for companies that adopt and adapt 
to it. However, convincing potential customers of the efficiency and impact of intralogistics robots presents challenges. 
These limitations include high training and implementation costs, changes to work routines, and the need for space and 
process adaptations. 


A recent study from CBRE, the world’s largest real estate services provider, revealed that European industrial and 
logistics investments increased by 16\% in Q1 of 2024 compared to Q1 of 2023. Despite this, many warehouses are old, 
repurposed buildings that are unorganized due to the nature of their daily tasks. These brownfield warehouses are 
expensive to maintain and digitalize but represent ideal grounds for developing and utilizing fully autonomous systems. 
Unlike AGVs, autonomous vehicles possess the intelligence and capability to plan and execute their plans efficiently. 
They are designed to adapt to uneven terrains and unorganized working environments given the revolutionary technologies 
that they hold. 

In this context, STILL, a KION group company, has been developing smart intralogistics solutions since 
its establishment more than a 100 years ago, successfully integrating automation into logistics. STILL offers a wide 
variety of products that cater to industries ranging from food retail to automotive manufacturing and chemical sectors. 
Their solutions address various customer challenges, such as reaching high shelves, order picking, palletizing, 
fleet management, and providing consulting services. Trusted by leading German companies like Siemens, STILL's products 
and services are renowned for their reliability and efficiency. 

\newpage
\thispagestyle{intro} 

The STILL Autonomous Robots department focuses on developing and enhancing smart vehicles. These autonomous robots, 
with minimal cost-effective input from the warehouse environment, can perceive their surroundings, estimating their 
positions, efficiently planning future tasks, controlling their movements to reach destinations, executing desired 
actions, and making corrections if necessary. This focus on smart, autonomous vehicles demonstrates STILL's commitment 
to pushing the boundaries of intralogistics and automation. 

In light of this, this thesis aims to contribute to the process of palletizing by optimizing a local path planning 
approach applied in the warehouse's stations near the shelves or spots where pallets are located for picking or in 
free placing areas. The developed approach seeks to plan the near-field path optimally while simultaneously avoiding 
obstacles. 

The objective is to create predictable, repeatable, and explainable vehicle behaviors, demonstrating the autonomous 
vehicle's ability to generate effective solutions tailored to each specific scenario. By focusing on optimal, 
pattern-based near-field path planning, this thesis addresses the challenge of navigating complex intralogistics 
environments, ensuring maximum efficiency and safety in operations. This approach not only enhances the vehicle's 
performance but also showcases the potential of autonomous technology in transforming modern intralogistics. 


This work encloses 4 chapters: 
\begin{itemize}
    \item \textbf{Chapter 1} gives a deep insight about the host company’s structure, activities and products. 
    Then it dives into the project context and it motivations, the studied problematic, the fundamental aspects of 
    the work, the thesis specifications and, the work methodology. 
    \item \textbf{Chapter 2} delves into the state of the art of the work area, then goes through a review of the 
    literature that served as a base of the thesis and gave an overview of the existing solutions. Finally, 
    it presents milestones followed in the course of the thesis work. 
    \item \textbf{Chapter 3} explains the development steps of the approach: it presents the mathematical aspect of 
    splines and their implementation in robotic path planning, explains the geometric division of the stations 
    into transition zones, discusses the studied path discrimination approaches, and finally it explores the optimization 
    approaches for the local path planning problem. 
    \item \textbf{Chapter 4} explicits the steps it takes to implement the developed approach in the RACK framework, 
    test them in the RACK simulation system, then on the automated vehicle, run different test scenarios and states 
    the obtained results. 
\end{itemize}

\newpage
\thispagestyle{intro}